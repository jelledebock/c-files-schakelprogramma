\documentclass{article}
\usepackage{listings}
\lstset{language=C++}
\title{Notities les 8}
\author{Jelle De Bock}
\begin{document}
\maketitle
\section{Voorbeeldklasse}
\begin{lstlisting}
class Student{
    public:
        Student();
        Student(int nr, const string &nm, int aantal = 0, 
            const int *ptn = 0);  
        //waarom &nm je wil voorkomen om grote waardes 
        //te moeten kopieren
        ~Student(){
           if (punten) delete[] punten; //is een soort free 
        }       
        void print() const; //een garantie dat je niets wijzigt 
        //in klasse

    private:
        int stnr;
        string naam;
        int aantalPunten;
        int *punten;
};

Student::Student()
{
    stnr = 0; //niet auto nul!!
    aantalPunten = 0;
    punten = 0;
}

void Student::print() const{
    //...
}

int main()
{
    Student s1(20021923, "Jan Jansen", 4, ptn); //constructor opgeroepen
    Student s3; //default constructor opgeroepen
}
\end{lstlisting}
\section{De copy constructor}
Kopieert een bestaand object naar een nieuw objectje. Wordt in cpp zeer vaak
gebruikt. Standaard is hij al aanwezig. Deze wordt immers vaak gebruikt,
bijvoorbeeld in volgende gevallen.
\begin{itemize}
    \item \texttt{Student std2(std1);}
    \item \texttt{return std1};
    \item \texttt{void functie(Student st) //st is value parameter}
\end{itemize}
\end{document}
