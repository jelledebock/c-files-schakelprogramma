\documentclass{article}
\usepackage{listings}
\lstset{language=C}
\title{Notities les 5}
\author{Jelle De Bock}
\begin{document}
\maketitle
\section{Const}
\begin{itemize}
    \item \textbf{Twee posities mogelijk} \\
            \textit{const int *t; /\/\ pointer naar een constante int}\\
            \textit{int const * t; /\/\ is identiek als bovenstaande, de bovenstaande wordt echter meer gebruikt  }
            \textit{int * const t; /\/\ constante pointer}\\
           Een pointer slaat altijd op het geen wat er voor komt, behalve wanneer het het helemaal vooraan staat, dan wat er net achter komt.
    \item \textbf{Kleine oefening hierop} \\
        \begin{lstlisting}
            const int * const ** a;
            //a is een pointer naar een pointer naar een constante
            //pointer naar een constante int
            //A->[]->[]=>[const int]
            a++; //oke
            (*a)++; //oke
            (**a)++; //neen, want constant
            (***a)++; //neen, want constante int
        \end{lstlisting}
    \item \textbf{Const bij arrays} \\
            Niet te ver gaan met const bij pointers. Je moet er enkel zeker van zijn dat de values niet gewijzigd worden.
\end{itemize}
\section{Structs}
   Tracht bij structs zo veel mogelijk met pointers te werken, want structs kunnen enorm groot zijn. 
\section{Stack en heap}
    Bij malloc plaats je iets op de heap. Bij stack wordt deze leeggemaakt op einde van de methode.
\section{Linked lists}
Nadeel van een array, is dat alles achter elkaar moet zitten. \\
Invoegen in een array is een tijdsintensieve taak.
\subsection{Structuur van de linked list}
Einde van de lijst wordt aangegeven door null pointer, tenzij het een circulaire lijst is.
\\
Dubbelgelinkte lijst, heeft ook pointer naar zijn vorige knoop.
\subsection{Een implementatiesuggestie}
    \begin{lstlisting}
        typedef struct knoop knoop;
        //dit doe je zodat je niet elke keer "struct"
        //hiervoor moet zetten (1x bij implementatie)
        struct knoop{
            int getal;
            knoop * next; 
        }
    \end{lstlisting}
\subsection{Toevoegen aan een gelinkte lijst}
    \textbf{Enkele situaties}
    \begin{itemize}
        \item \textbf{Achteraan toevoegen} (zie \textit{oefeningen_knoop.c} file)
    \end{itemize}
\subsection{Wijzigen}
    Om iets te wijzigen moet je met een pointer naar een pointer werken.
\end{document}
