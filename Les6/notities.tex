\documentclass{article}
\usepackage{listings}
\lstset{language=C}
\title{Notities les 5}
\author{Jelle De Bock}
\begin{document}
\maketitle
\section{Recap vorige les c++}
Referentietype: \textit{[const] type  \&var} \\
eigenlijk wordt het adres meegegeven, is perfect om te gebruiken bij methodes
in C++. Ook zeker const gebruiken wanneer het kan. \\
Console invoer en uitvoer: \textit{cin en cout}. Je kan ze concateneren.
\begin{lstlisting}
    cout << "Het getal is" << getal; //uitschrijven
    cin >> naam;
    getline(cin, naam);
\end{lstlisting}
Getline na cin kuist de lijn van cin op (de enter).
Get() vs getchar() get aanvaardt ook whitespaces...
\section{Namespace}
Erg vergelijkbaar met packages in Java. Dienen om collisions tussen functies
met dezelfde naam uit verschillende libs te vermijden.
Verschillende manieren van oproeping:
\begin{lstlisting}
    using std:string;
    using std::cout;
    using namespaces std;
\end{lstlisting}
Opgelet wanneer er nog eens een functie met dezelfde naam in uw file staat. Dan
kan de compiler niet weten de welke je bedoelt (die uit de package of die in de
file).
\section{ios package}
Een package die schrijven van / naar files mogelijk maakt.
\subsection{Oefeningetje}
De gebruiker mag 5 lijnen tekst ingeven. Deze weorden toegevoegd achteraan een
bestand.

\begin{lstlisting}
    #include <fstream>
    using namespace std;

    int main()
    {
        string lijn;
        ofstream bestand("tekst.txt",ios::app);

        for(int i=0;i<5;i++)
        {
            getline(cin,lijn);
            bestand << lijn << endl;
        }
        bestand.close();
    }   
\end{lstlisting}

\subsection{slide 37}
Wat doe het programmafragment?\\
Bestand openen voor lezen en schrijven. \\
Lees getal in tot er geen meer staat, \\
schrijf het gewijzigde getal weg. MAAR: schrijven en lezen tegelijkertijd is
dikke miserie...

\subsection{Opmerkingen}
Telkens je een ifstream wil gebruiken moet je de referentieoperator gebruiken.

\section{Malloc en calloc varianten}
Je mag \textit{malloc} en \textit{calloc} en de daarbijhorende \textit{free}
nog gebruiken, maar je kan nu ook de modernere \textit{new} en \textit{delete}
gebruiken. Voor arrays gebruik je \textit{new[]} en \textit{delete[]}

\end{document}
